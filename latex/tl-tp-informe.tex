\documentclass[11pt]{article}
\usepackage{caratula}

\usepackage[utf8]{inputenc}
\usepackage[spanish,activeacute]{babel}

\usepackage{lmodern}
\usepackage[T1]{fontenc}

\usepackage{amsmath,amssymb,amsthm,mathabx,mathtools}

% ----------------------------------------------------------------------------

\materia{Teoría de Lenguajes}
\submateria{Primer cuatrimestre de 2017}
\titulo{Analizador sintáctico y semántico para $\lambda^{bn}$}
\subtitulo{Trabajo práctico}
\grupo{[FALTA NOMBRE DEL GRUPO]}

\integrante{Franco Frizzo}{013/14}{francofrizzo@gmail.com}
\integrante{Iván Pondal}{078/14}{ivan.pondal@gmail.com}
\integrante{Alfredo Sanzo}{-}{alfredo.sanzo@gmail.com}

% ----------------------------------------------------------------------------

\usepackage[a4paper,
            left=2.53cm,
            right=2.3cm,
            top=3.2cm,
            bottom=2.5cm]{geometry}
\usepackage{xcolor}
\usepackage[colorlinks=true,
            linkcolor={blue!75!black},
            urlcolor={blue!75!black},
            bookmarks=true]{hyperref}

\setlength\parskip{.3em}
\setlength\headsep{1.25em}
\renewcommand\arraystretch{1.25}

\usepackage{enumitem}
\setenumerate{itemsep=.2em,topsep=.2em}
\setitemize{itemsep=.2em,topsep=.2em}

\usepackage{fancyhdr}
\usepackage{lastpage}
\pagestyle{fancy}
\lhead{\Materia}
\rhead{\Titulo}
\cfoot{{\thepage} de \pageref{LastPage}}

% ============================================================================

\begin{document}

% ----------------------------------------------------------------------------

\thispagestyle{empty}
\maketitle

\newpage
\newpage
\tableofcontents

\newpage

% ============================================================================

\section{Introducción}

En este trabajo, presentamos la implementación de un analizador léxico,
sintáctico y semántico para el cálculo lambda tipado con valores de verdad
(\emph{booleanos}) y números naturales ($\lambda^{bn}$).

Para ello utilizamos la herramienta \textsc{PLY}(Python Lex-Yacc) que dada
el conjunto de tokens y gramática de atributos del lenguaje a implementar,
 genera un parser \textsc{LALR} para la misma.

\section{Formalización del lenguaje}

\subsection{Analizador léxico: \emph{tokens}}



\subsection{Analizador sintáctico: gramática}

\begin{tabular}{rrl}
$S$  & $\rightarrow$ \qquad & $E$ \\
$E$  & $\rightarrow$ \qquad & \verb|if |$E$\verb| then |$E$\verb| else |$E$ \\
     & $\vert$              & \verb|\var:|$T$\verb|.|$E$ \\
     & $\vert$              & $A$ \\
$A$  & $\rightarrow$ \qquad & $A$\verb| |$E'$ \\
     & $\vert$              & $E'$ \\
$E'$ & $\rightarrow$ \qquad & \verb|(|$E$\verb|)| \\
     & $\vert$              & \verb|succ(|$E$\verb|)| \\
     & $\vert$              & \verb|pred(|$E$\verb|)| \\
     & $\vert$              & \verb|iszero(|$E$\verb|)| \\
     & $\vert$              & \verb|0| \\
     & $\vert$              & \verb|nat| \\
     & $\vert$              & \verb|bool| \\
     & $\vert$              & \verb|var| \\
$T$  & $\rightarrow$ \qquad & $T'$\verb|->|$T$ \\
     & $\vert$              & $T'$ \\
$T'$ & $\rightarrow$ \qquad & \verb|type| \\
     & $\vert$              & \verb|(|$T$\verb|)| \\
\end{tabular}

% ============================================================================

\section{Implementación}



% ============================================================================

\section{Casos de prueba}

% ============================================================================

\section{Manual del usuario}

% ============================================================================

\section{Conclusiones}

% ============================================================================

\end{document}